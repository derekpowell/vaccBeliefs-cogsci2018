% Template for Cogsci submission with R Markdown

% Stuff changed from original Markdown PLOS Template
\documentclass[10pt, letterpaper]{article}

\usepackage{cogsci}
\usepackage{pslatex}
\usepackage{float}
\usepackage{caption}

% amsmath package, useful for mathematical formulas
\usepackage{amsmath}

% amssymb package, useful for mathematical symbols
\usepackage{amssymb}

% hyperref package, useful for hyperlinks
\usepackage{hyperref}

% graphicx package, useful for including eps and pdf graphics
% include graphics with the command \includegraphics
\usepackage{graphicx}

% Sweave(-like)
\usepackage{fancyvrb}
\DefineVerbatimEnvironment{Sinput}{Verbatim}{fontshape=sl}
\DefineVerbatimEnvironment{Soutput}{Verbatim}{}
\DefineVerbatimEnvironment{Scode}{Verbatim}{fontshape=sl}
\newenvironment{Schunk}{}{}
\DefineVerbatimEnvironment{Code}{Verbatim}{}
\DefineVerbatimEnvironment{CodeInput}{Verbatim}{fontshape=sl}
\DefineVerbatimEnvironment{CodeOutput}{Verbatim}{}
\newenvironment{CodeChunk}{}{}

% cite package, to clean up citations in the main text. Do not remove.
\usepackage{cite}

\usepackage{color}

% Use doublespacing - comment out for single spacing
%\usepackage{setspace}
%\doublespacing


% % Text layout
% \topmargin 0.0cm
% \oddsidemargin 0.5cm
% \evensidemargin 0.5cm
% \textwidth 16cm
% \textheight 21cm

\title{Articulating lay theories through graphical models: A study of beliefs
surrounding vaccination decisions}


\author{{\large \bf Derek Powell} \\ \texttt{derekpowell@stanford.edu} \\ Department of Psychology \\ Stanford University \And {\large \bf Kara Weisman} \\ \texttt{kweisman@stanford.edu} \\ Department of Psychology \\ Stanford University \And {\large \bf Ellen M. Markman} \\ \texttt{markman@stanford.edu} \\ Department of Psychology \\ Stanford University}

\begin{document}

\maketitle

\begin{abstract}
How can we leverage the cognitive science of lay theories to inform
interventions aimed at correcting misconceptions and changing behaviors?
Focusing on the problem of vaccine skepticism, we identified a set of 14
beliefs we hypothesized would be relevant to vaccination decisions. We
developed reliable scales to measure these beliefs across a large sample
of participants (\emph{n} = 1130) and employed state-of-the-art
graphical structure learning algorithms to uncover the relationships
among these beliefs. This resulted in a graphical model describing the
system of beliefs relevant to childhood vaccinations, with beliefs
represented as nodes and their interconnections as directed edges. This
model sheds light on how these beliefs relate to one another and can be
used to predict how interventions aimed at specific beliefs will play
out across the larger system. Moving forward, we hope this model will
help guide the development of effective, theory-based interventions
promoting childhood vaccination.

\textbf{Keywords:}
graphical modeling; lay theories; conceptual change; behavioral
interventions
\end{abstract}

Much of the richness of human thought depends on our ability to combine
and synthesize information into coherent belief systems, lay theories,
and mental models. These cognitive processes are vital for interpreting,
explaining, and predicting events; and for planning actions to intervene
on the course of these events. But these same abilities can sometimes
lead people astray, generating misconceptions that result in
inappropriate and even dangerous actions. Here, we focus on one striking
and timely example: The resurgence of diseases like measles in the wake
of widespread misconceptions about the safety of childhood vaccines.

In a larger project, we aim to develop effective ways to address this
and other misconceptions by leveraging the cognitive science of lay
theories to effect conceptual and behavioral change (see Weisman \&
Markman, 2017, for a review of this approach). In this paper, our goal
is to enrich our understanding of the conceptual ``ecosystem'' that
supports or discourages vaccination. To this end, we develop a graphical
model that describes the system of beliefs relevant to childhood
vaccinations, representing these beliefs as nodes and their
interconnections as directed edges. Moving forward, we hope the use of
these formal techniques will let us make quantitative inferences and
predictions to help guide the development of educational interventions.

\subsection{Vaccine beliefs and
misconceptions}\label{vaccine-beliefs-and-misconceptions}

In the early 2000s, now-discredited research led many people to believe
that childhood vaccinations, such as the Measles, Mumps, and Rubella
(MMR) vaccine, could increase children's risk for autism. Vaccination
rates declined in many communities, leading to a resurgence of
preventable childhood diseases: In 2014 the CDC tracked 667 cases of
measles in the US, where the disease had previously been eradicated
(CDC, 2015). Vaccines do not, in fact, cause autism (Taylor, Swerdfeger,
\& Eslick, 2014), but these misconceptions have proved to be remarkably
difficult to correct (e.g., Betsch \& Sachse, 2013; Horne, Powell,
Hummel, \& Holyoak, 2015).

One challenge to addressing misconceptions is that they are often
embedded in larger, internally coherent belief systems that guide how
people interpret and respond to evidence (Lewandowsky, Ecker, Seifert,
Schwarz, \& Cook, 2012). Suppose one thinks the infant immune system is
immature, weak, and easily overwhelmed: It might then seem unreasonable
to vaccinate a 2-month-old baby against 5 or more diseases at once, as
the CDC recommends. Similarly, if someone believes that the medical
community is unduly influenced by pharmaceutical companies, she might be
skeptical when medical studies come out in favor of these companies'
interests. Such beliefs might sustain the misconception that vaccinating
children is dangerous, even in the face of counter-evidence.

For educational interventions to be effective, they must be sensitive to
the broader conceptual context in which they'll be interpreted. In the
case of vaccine attitudes, interventions that simply emphasize the
safety of vaccines may not be convincing to people who hold strong
beliefs about the vulnerability of the infant immune system or
corruption in medical research---but other beliefs might be more
amenable to revision. Consistent with this, Horne et al. (2015) found
that straightforward reassurances of vaccine safety were ineffective in
changing people's attitudes toward vaccination, but informing people
about the risks of measles, mumps, and rubella resulted in more positive
views of childhood vaccination. As in many domains, our understanding of
the conceptual system driving vaccination decisions is limited. Having
hypothesized that some set of beliefs might be relevant to people's
vaccination decisions, it would be extremely useful to validate these
intuitions and specify precisely how these beliefs relate to or inform
one another.

How can we effectively transform a qualitative account into a useful,
testable, model of a lay theory? In this paper, we describe a graphical
modeling approach to developing a rich, formal theory of the beliefs
surrounding vaccination decisions. We began by identifying potentially
relevant beliefs, developing reliable instruments for measuring them,
and using those instruments to survey a large sample of participants.
Then, we used a state-of-the-art graphical modeling approach---Bayesian
network structure learning (for a review, see Scutari \& Denis,
2014)---to discover and describe connections among these beliefs and
represent them in a quantitative model. We consider this project a first
step in a longer process that, we hope, will yield a rich,
quantitatively precise theory of this conceptual system.

\section{Study}\label{study}

Our goal was to use behavioral data to develop a graphical model of a
conceptual system that could support or discourage vaccination. This
process involves many choices about data representation, as well as
trade-offs between the fit, complexity, and intelligibility of the
models produced. Here we describe the steps we took to build this model,
highlighting key decision points that shaped the resulting model.

\subsection{Scale development: Identifying and measuring relevant
beliefs}\label{scale-development-identifying-and-measuring-relevant-beliefs}

The first and perhaps most consequential decision points concern the set
of beliefs we hypothesized are relevant to vaccination decisions, and
how we chose to measure these beliefs. Our outcome of interest was
participants' intentions to vaccinate their children (\emph{vaccination
intentions}). Drawing on a variety of sources, including research on
anti-vaccine skepticism, anti-vaccine websites, and our own qualitative
surveys with vaccine skeptics (not reported here), we generated a list
of 13 underlying beliefs that might influence this outcome.

These included two broad worldviews: (1) \emph{naturalism}, a general
preference for natural over artificial things; and (2) \emph{holistic
balance}, one important aspect of attitudes toward alternative medicine
(McFadden, Hernández, \& Ito, 2010), as well as three slightly more
specific theories about parenting and medicine: (3) general
\emph{parental protectiveness}; (4) \emph{parental expertise}, namely
the belief that parents usually know more about their children's health
than medical experts; and (5) \emph{medical skepticism}, including
concerns about pharmaceutical companies and corruption in the medical
community. In addition, we identified a variety of specific claims about
vaccines that seemed important to people's arguments for and against
vaccination, including beliefs about (6) the overall safety of vaccines
(\emph{vaccine danger}); (7) \emph{toxic additives in vaccines}; and (8)
\emph{vaccine effectiveness}, how effective vaccines are in preventing
disease; as well as a variety of specific claims about childhood
diseases like measles, mumps, and rubella, including beliefs about (9)
\emph{disease rarity} and (10) \emph{disease severity}. Beyond this, we
theorized that intuitive theories of the infant immune system might be
relevant, including beliefs that (11) the infant immune system is weak
(\emph{IIS: weakness}); (12) the infant immune system is limited in its
capacity and can be easily overwhelmed (\emph{IIS: limited capacity});
and (13) vaccines strain the infant immune system (\emph{IIS: vaccines
strain}).

We then developed psychometrically robust scales to measure these
beliefs, stipulating that each scale should be brief, composed of 4-6
statements for participants to evaluate; include at least one
reverse-coded item; and be highly reliable (Cronbach's \(\alpha \geq\)
.80). After extensive piloting and refinement, we created 14 scales that
met these criteria, including one preexisting scale (the ``holistic
balance'' subscale from McFadden et al., 2010). Final observed
reliability ranged from .73 to .91. (A full list of items for all scales
is available at \url{https://osf.io/dc5j8/}.)

\subsection{Method}\label{method}

To investigate relationships among the beliefs surrounding vaccination
intentions, we examined covariation among these beliefs across a large
sample of participants. For instance, if someone strongly endorses
medical skepticism, how might this influence their beliefs about the
toxicity of vaccines, or the severity of diseases like measles? These
observed covariation relationships shed light on how these beliefs hang
together and influence one another and, combined with structure learning
algorithms, provide a path toward approximating this conceptual system.

\subsubsection{Participants}\label{participants}

1200 people participated via Amazon Mechanical Turk. All participants
had gained approval for \(\geq\) 95\% of previous work (\(\geq\) 100
assignments); had verified US MTurk accounts; and indicated that they
were \(\geq\) 18 years old. Participants were paid \$1.60 for about 8
minutes of their time. Repeat participation was prevented.

\subsubsection{Procedure}\label{procedure}

Participants were told that we were interested in their opinions about a
variety of topics. They then proceeded through our 14 scales, rating
each statement on a scale from ``Strongly disagree'' (coded as -3) to
``Strongly agree'' (+3); the order of presentation of these scales and
the order of questions within each scale was randomized for each
participant.

Two attention checks (e.g., ``Please select somewhat agree'') were
embedded randomly among these questions; the 70 participants who failed
at least one of these checks were excluded from further analyses. This
left a final sample of \emph{n} = 1130 (94\% of our full sample).

\subsubsection{Data preparation}\label{data-preparation}

Scores for each scale were calculated as the average of the responses to
questions in that scale, after reverse-coding; for all scales, the
theoretical range of scores was -3 to +3. The final dataset for modeling
included 14 scores for each participant.

\section{Model Building}\label{model-building}

Our primary goal was to build a formal model that could approximate the
conceptual relationships among beliefs related to vaccine intentions.
This bring us to our second decision point: How to model the data. We
conceived of these beliefs as influencing one another, where influences
are directed from one belief to another as in, for example, causal
relationships (Pearl, 2000) and logical implication (Williamson, 2001).
These types of asymmetric relations can be well-captured in a Directed
Acyclic Graph (DAG), where each belief is represented as a node in a
network, and all edges between nodes are directed, i.e., connections run
in one direction only. For instance, an edge from naturalism to medical
skepticism would indicate that naturalism beliefs influence medical
skepticism. Because we measured beliefs continuously, we employed
gaussian (linear) DAGs.

This class of models has several desirable qualities. First, there are
efficient algorithms for ``learning'' these network structures from
data, allowing us to discover possible relationships among beliefs using
observed correlations in a large sample of participants. Second, a DAG
can be used to generate inferences based on information about a subset
of the network's nodes. This allows us to predict a person's beliefs
about a given topic (e.g., vaccine safety) based on observations of
their beliefs about another topic (e.g., medical skepticism). Finally,
these networks are capable of generating predictions about the
consequences of intervening on nodes within these systems, an important
advantage when using these networks to craft real-world interventions.

\begin{CodeChunk}
\begin{figure*}[h]

{\centering \includegraphics{figs/cv_plot-1} 

}

\caption[Cross-validation results]{Cross-validation results. Left: Log-likelihood loss predicting out-of-sample data across 10 run 10-fold cross-validation. Right: Number of edges in models generated by each algorithm. Algorithms are named according to the use of the theory-based blacklist, and the threshold used (e.g., “mmhc-theory-05” is the MMHC algorithm with the theory-based blacklist and $\alpha$ = .05).}\label{fig:cv_plot}
\end{figure*}
\end{CodeChunk}

\subsection{Incorporating theory}\label{incorporating-theory}

Structure-learning algorithms operate in a ``bottom-up'' fashion,
generating a model based on raw data. Still, there are opportunities to
exert ``top-down'' influences on this theory-building process. This
brings us to our third decision point: whether and how to constrain the
search for the structure connecting these beliefs. By ``whitelisting''
or ``blacklisting'' connections between nodes, we can stipulate that
they must or must not be included in the final model. Such constraints
could be specific (e.g., a link from A to B must be included) or broad
(e.g., C has no ``parents,'' i.e., no incoming connections; D has no
``children,'' i.e., does not feed into any other nodes).

Before constructing our model, we sorted the 14 measured beliefs into
``tiers'' based on how broad or abstract each belief was. For instance,
we considered \emph{holistic balance} and \emph{naturalism} to be the
most abstract beliefs measured, and labeled these ``worldviews''; we
considered our outcome of interest, \emph{vaccine intentions}, to be the
most concrete measurement of a specific ``intention.'' Figure 3 shows
the level assigned to each node in the network. We used this hierarchy
of beliefs to induce a blacklist that would constrain our search space.
We made the assumption that the beliefs surrounding vaccine decisions
would be best described as a generative model, in which more abstract
beliefs set expectations for more concrete beliefs or observations
(following, e.g., Jern, Chang, \& Kemp, 2014). In other words,
``worldviews'' could feed directly into ``theories,'' ``claims,'' or
``intentions,'' but none of these more concrete beliefs could feed into
``worldviews''; likewise, ``theories'' could feed into ``claims'' or
``intentions'' (but not vice versa); and ``claims'' could feed into
``intentions'' (but not vice versa). This approach offers a highly
generalizable means to incorporating existing a priori theories into the
structure learning process.

\begin{CodeChunk}
\begin{figure*}[h]

{\centering \includegraphics{figs/validation_plot-1} 

}

\caption[Observed versus predicted values for each belief in the testing set, with predictions from the final BDHC model using posterior probability threshold = .95 and fit to the training split]{Observed versus predicted values for each belief in the testing set, with predictions from the final BDHC model using posterior probability threshold = .95 and fit to the training split.}\label{fig:validation_plot}
\end{figure*}
\end{CodeChunk}

\subsection{Structure learning
algorithms}\label{structure-learning-algorithms}

We now turn to our fourth decision-point, the selection of a
structure-learning algorithm. Here, we consider two structure learning
algorithms implemented in the bnlearn R package (v4.2)-- the score-based
hill climbing (HC) algorithm and the hybrid Min-Max Hill Climbing (MMHC)
algorithm (Scutari, 2010). In addition, we introduce our own hybrid
approach that may offer some appealing qualities for our purposes. Our
approach is similar to MMHC, which first restricts the search space for
a directed graph by finding an undirected ``skeleton'' describing
conditional-independence relationships among variables. However, unlike
the MMHC algorithm, which uses the ``min-max parents'' (MM) heuristic
algorithm to constrain the search space, we use state-of-the-art
Bayesian structure learning algorithms implemented in the BDgraph R
package (Mohammadi \& Wit, 2017) to identify this undirected skeleton.
Like MMHC, our approach then uses the HC algorithm to find a directed
graph. We will refer to this custom algorithm as ``BDHC.''

\subsection{Achieving intelligibility}\label{achieving-intelligibility}

Because we aim to develop interventions based on the theory emerging
from our model, an important desiderata for this model is
intelligibility. This raises a fifth decision point: the degree to which
we are willing to trade off predictive accuracy in exchange for greater
intelligibility.

Some degree of simplicity is likely necessary for intelligibility. One
proxy for simplicity is sparsity, or the number of edges present in the
graph. Both MMHC and our custom algorithm, BDHC, offer a fairly direct
means to impose varying degrees of sparsity on the resulting graph. In
MMHC the modeler is free to choose the (frequentist) \(\alpha\)
criterion for the restriction phase: A higher \(\alpha\) value results
in fewer edges. Similarly, using BDHC the modeler can set the threshold
for the posterior probability of edges to be included in the skeleton:
In this approach, edges are present in the final graph only when the
posterior probability that there is a dependency between these nodes,
independent of the other variables, is greater than some specified
threshold (e.g., .95).

\subsection{Cross-validation and algorithm
selection}\label{cross-validation-and-algorithm-selection}

We have highlighted five key decision points in constructing our model.
Several of these, including choosing an algorithm and a threshold for
retaining edges, can be aided by empirical cross-validation procedures,
which allow us to explore a large space of models while avoiding
overfitting. With this in mind, we split our data into a ``training
split'' (80\% of the data), which we used to develop and compare models,
and a ``testing split'' (20\%), which we used to validate the final
model's performance. We performed 10 runs of 10-fold cross-validation on
the training data to compare the performance of our different
approaches, using identical fold-splits for all models. Using this
procedure, we compared the HC, MMHC, and BDHC algorithms, using various
values for alpha (MMHC) and posterior thresholds (BDHC), and including
or omitting our theory-based blacklist.

Cross-validation results comparing these models are shown in Figure 1.
We were interested both in how well the models produced by these
algorithms performed in an out-of-sample prediction (as indexed by their
log likelihood loss) as well as in how complex the resulting models were
(as indexed by the number of edges in the resulting graphs).

A few points are apparent from the results of cross-validation. First,
the inclusion of the theory-based blacklist (``tiers'' of abstractness)
had relatively little impact on model performance. This is promising, as
it suggests our existing theory is not in conflict with the data.

Second, there is a trade-off between the degree to which the algorithm
is tuned toward sparsity and the resulting fit, such that more complex
models generally provide somewhat better fits. If we were prioritizing
predictive power, we would proceed with the best-fitting model
(HC-theory); if we were prioritizing simplicity, we might opt to proceed
with the sparsest model (BDHC-theory-99). For the purposes of designing
real-world interventions, we would like a model that both allows us to
make accurate out-of-sample predictions and that provides an
intelligible theory. Striking the ``right'' balance between predictive
power and intelligibility is difficult to resolve formally.

We thus proceeded informally, attempting to balance concerns for fit and
intelligibility in proportion to our project's goals. Averaging across
folds, the likelihood ratio of observed data under the best-fitting
model (HC-theory) compared with the worst-fitting model (BDHC-theory-99)
was only 1.19. Although reliable, these differences in fit are not
sufficient to motivate adopting the most complex models. Instead, we
sought to identify the best-fitting model that was sufficiently simple
and intelligible for our purposes. To assess intelligibility more
directly, we used each algorithm to learn a graph based on the entire
set of training data (n = 904). From among these different options, we
chose to proceed with the model resulting from the BDHC method with a
posterior probability threshold of .95.

This resulted in a partially-directed acyclic graph (PDAG) with three
undirected edges. To generate model predictions for validation, we chose
to set these edge directions arbitrarily, under the assumption that they
will not meaningfully impact prediction performance due to
score-equivalence (Scutari \& Denis, 2014). The final resulting network
is shown in Figure 3.

\subsection{Validating the model's
performance}\label{validating-the-models-performance}

To evaluate the model's performance, we tested its accuracy in
predicting responses among the remaining 20\% testing split (\emph{n} =
226). After learning the network and fitting its parameters using the
training data split, we generated predictions for held-out participants'
responses for each variable by conditioning the network on the remaining
13 (observed) variables. Figure 2 compares the model's predictions with
participants' actual responses.

Collapsing across all variables, the average correlation between
predicted and observed responses was \emph{r} = .825, accounting for
68.1\% of the variance in observed responses. Correlations between
observed and predicted values ranged from .304 to .899 across the
different belief scales. In general, the model shows greater predictive
accuracy for more central beliefs (e.g., \emph{vaccine danger}) than for
more distant beliefs (e.g., \emph{parental protectiveness}). Altogether,
this out-of-sample predictive performance suggests this model can
usefully predict and explain participants' beliefs.

\begin{CodeChunk}
\begin{figure*}[h]

{\centering \includegraphics[width=1\linewidth]{figs/network_result_plot-1} 

}

\caption[Final BDHC model using posterior probability threshold = .95]{Final BDHC model using posterior probability threshold = .95. Nodes are labeled for abstractness, from worldviews (w), to theories (t), claims (c), and intentions (i). Edge weights indicate standardized linear coefficients from the gaussian model, which can be interpreted as regression coefficients. Asterisks indicate edges that were directed arbitrarily.}\label{fig:network_result_plot}
\end{figure*}
\end{CodeChunk}

\section{Discussion}\label{discussion}

We developed a graphical model of a conceptual ``ecosystem'' surrounding
vaccination decisions, by combining an initial qualitative theory with
behavioral data using Bayesian network structure learning. The resulting
model (Figure 3) offers a preliminary description of the conceptual
systems that support and discourage vaccination decisions.

The ultimate value of this model rests heavily on its validation by
future interventional studies. With that in mind, we consider some
preliminary insights and implications. First, this model confirms that
the 14 beliefs we hypothesized would be relevant to vaccine decisions
are, in fact, closely related to each other and to participants'
intentions to vaccinate their children. Many of the conceptual
connections revealed by this model make sense intuitively. For example,
beliefs about the effectiveness of vaccines, the safety or danger of
vaccines, and the severity of childhood diseases are the three nodes
with direct connections to \emph{vaccination intentions}. Such findings
provide one check on the success of the model-building process, and
suggest it is uncovering meaningful relationships.

Other findings may shed new light on the role of lay theories in vaccine
decisions. For example, a ``naturalist'' worldview---the general view
that natural things are better than artificial things---appears to be
strongly related to medical skepticism and parental expertise; all three
of these abstract beliefs are related to concrete beliefs that, in turn,
feed into participants' vaccination intentions. This finding supports
some of our earlier speculations as to why interventions have often
failed to alter vaccine skepticism: These beliefs may be tied into
far-ranging worldviews that affect many aspects of people's thinking,
including their interpretation and response to evidence about the safety
of vaccines.

Finally, the current model highlights certain beliefs that might be
especially influential in shaping vaccination decisions, such as beliefs
about \emph{naturalism}, \emph{vaccine danger}, \emph{vaccine
effectiveness}, and \emph{toxic additives in vaccines}. Of course, some
of these beliefs may be more or less amenable to interventions. For
instance, previous work suggests that it may be difficult to craft
interventions that effectively dispel beliefs about \emph{vaccine
danger} (e.g., Horne et al., 2015). Still, by revealing the
interconnections among these beliefs, the model suggests ways to
overcome these challenges. One promising approach could be to combine
successful interventions from past research, such as providing
information about the severe dangers of diseases like measles for
infants and young children (Horne et al., 2015), with information about
how and why vaccines work so well to protect children from these
diseases (targeting \emph{vaccine effectiveness}).

Conversely, some interventions that initially seemed promising now seem
more complicated. For example, we initially hoped that providing
information to parents about how the infant immune system works---in
particular, dispelling the misconception that it has a limited
capacity---could promote positive attitudes toward vaccination. We were
disappointed to observe the weak first-order correlation between this
belief and vaccine intentions in our behavioral data (\emph{r} = -.097
in our training split). The model sheds light on this surprising (lack
of) relationship: Although the belief that the infant immune system is
limited in capacity is positively related to the belief that vaccines
strain the immune system---discouraging vaccination, as we had
assumed---it also seems to promote the belief that childhood diseases
have severe consequences for young children, which might, in turn,
\emph{encourage} vaccination. In light of this, we speculate that
attempting to dispel beliefs about limited capacity might have no effect
on a person's vaccine intentions (due to these countervailing
forces)---or such an intervention might have different effects for
different people, depending on their auxiliary beliefs (e.g., about
disease severity). Simulation studies using this model could help
elucidate these possibilities, and will be critical as we continue to
pursue effective interventions.

Moving forward, we envision an iterative process in which we continue to
combine bottom-up, data-driven insights with top-down theorizing to
refine our understanding and develop effective interventions. First, we
can use the model to simulate how interventions targeting specific
beliefs or combinations of beliefs will affect beliefs throughout the
wider network. Based on these predictions, we can choose optimal sites
of intervention, craft interventions aimed at changing these target
beliefs, and measure their effects. Studies and simulations will allow
us to identify where the model succeeds or fails, and revise our model
and theory accordingly (e.g., by reversing the direction of edges,
adding missing variables, specifying interactions, or modeling
non-linear relationships). If these interventions have positive
outcomes, we can begin translating them into more applied contexts.

Developing educational interventions is difficult, and testing these
interventions, particularly in person, can be extremely costly. Here, we
illustrated a promising and novel method for moving effectively from
intuitions about lay theories to empirically validated methods for
correcting misconceptions and improving decisions.

\section{References}\label{references}

\setlength{\parindent}{-0.1in} \setlength{\leftskip}{0.125in} \noindent

\hypertarget{refs}{}
\hypertarget{ref-Betsch2013}{}
Betsch, C., \& Sachse, K. (2013). Debunking vaccination myths: strong
risk negations can increase perceived vaccination risks. \emph{Health
Psychology}, \emph{32}(2), 146--55.

\hypertarget{ref-CDC2015}{}
Centers for Disease Control and Prevention. (2015). Measles in the us.
Retrieved from
\url{http://www.cdc.gov/media/releases/2015/t0129-measles-in-us.html}

\hypertarget{ref-Horne2015}{}
Horne, Z., Powell, D., Hummel, J. E., \& Holyoak, K. J. (2015).
Countering antivaccination attitudes. \emph{Proceedings of the National
Academy of Sciences of the United States of America}, \emph{112}(33),
10321--4.

\hypertarget{ref-Jern2014}{}
Jern, A., Chang, K.-M. K., \& Kemp, C. (2014). Belief polarization is
not always irrational. \emph{Psychological Review}, \emph{121}(2),
206--224.

\hypertarget{ref-Lewandowsky2012}{}
Lewandowsky, S., Ecker, U. K. H., Seifert, C. M., Schwarz, N., \& Cook,
J. (2012). Misinformation and Its Correction: Continued Influence and
Successful Debiasing. \emph{Psychological Science in the Public
Interest}, \emph{13}(3), 106--131.

\hypertarget{ref-McFadden2010}{}
McFadden, K. L., Hernández, T. D., \& Ito, T. A. (2010). Attitudes
toward complementary and alternative medicine influence its use.
\emph{EXPLORE: The Journal of Science and Healing}, \emph{6}(6),
380--388.

\hypertarget{ref-Mohammadi2017}{}
Mohammadi, A., \& Wit, E. C. (2017, August). BDgraph: Bayesian structure
learning in graphical models using birth-death mcmc. R package version
2.40.

\hypertarget{ref-Pearl2000}{}
Pearl, J. (2000). \emph{Causality: Models, reasoning, and inference}.
New York, NY, USA: Cambridge University Press.

\hypertarget{ref-Scutari2010}{}
Scutari, M. (2010). Learning Bayesian Networks with the bnlearn R
Package, \emph{35}(3).

\hypertarget{ref-Scutari2014}{}
Scutari, M., \& Denis, J.-B. (2014). \emph{Bayesian networks with
examples in R}. Boca Raton: Chapman \& Hall.

\hypertarget{ref-Taylor2014}{}
Taylor, L. E., Swerdfeger, A. L., \& Eslick, G. D. (2014). Vaccines are
not associated with autism: An evidence-based meta-analysis of
case-control and cohort studies. \emph{Vaccine}, \emph{32}(29),
3623--3269.

\hypertarget{ref-Weisman2017}{}
Weisman, K., \& Markman, E. M. (2017). Theory-based explanation as
intervention. \emph{Psychonomic Bulletin and Review}, \emph{24}(5),
1555--1562.

\hypertarget{ref-Williamson2001}{}
Williamson, J. (2001). Bayesian networks for logical reasoning.
\emph{AAAI Technical Report}, 136--143. Retrieved from
\url{http://kar.kent.ac.uk/7396/}

\end{document}
